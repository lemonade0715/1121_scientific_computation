\documentclass[12pt]{article}
\usepackage{amsmath, amsthm, amssymb}
\newtheorem{lemma}{Lemma}

\begin{document}


%                           theorem2.1.6

%%%%                    induced matrix norm
\begin{proof}
    For any nonzero $\mathbf{x} \in \mathbb{C}^n$, \[\frac{\Vert A\mathbf{x} \Vert_\beta}{\Vert \mathbf{x} \Vert_\alpha} \leq \sup_{\mathbf{x} \neq 0}\frac{\Vert A\mathbf{x} \Vert_\beta}{\Vert \mathbf{x} \Vert_\alpha}=\Vert A \Vert_{\alpha,\beta}\]
    we have 
    \begin{equation}\label{equation1}
        \Vert A\mathbf{x} \Vert_\beta \leq \Vert A \Vert_{\alpha,\beta} \Vert \mathbf{x} \Vert_\alpha.
    \end{equation}
    Given $A,B \in \mathbb{C}^{n \times n}$, for any $x \in \mathbb{C}^n$ with $\Vert \mathbf{x} \Vert_\alpha=1$, by equation \ref{equation1}
    \[\Vert AB\mathbf{x} \Vert_\beta \leq \Vert A \Vert_{\alpha,\beta} \Vert B \mathbf{x}\Vert_\alpha \leq \Vert A \Vert_{\alpha,\beta}\Vert B \Vert_{\alpha,\beta} \Vert \mathbf{x} \Vert_\alpha. \]
    Therefore $\Vert A \Vert_{\alpha,\beta}\Vert B \Vert_{\alpha,\beta} \Vert \mathbf{x} \Vert_\alpha$ is an upper bound for $\Vert AB\mathbf{x} \Vert_\beta$.
    Moreover, 
    \[\Vert AB \Vert_{\alpha,\beta}=\sup_{\Vert \mathbf{x} \Vert_\alpha= 1}{\Vert AB\mathbf{x} \Vert_\beta} \leq \Vert A \Vert_{\alpha,\beta}\Vert B \Vert_{\alpha,\beta}\]
    Hence, the induced matrix norm is submultiplicative.
\end{proof}


%                            Frobenius norm
\begin{proof}
    Note that $A \in \mathbb{C}^{m \times n}$
    \[\Vert A\Vert_F^2=\sum_{i=1}^m \Vert A_i \Vert_2^2=\sum_{j=1}^n \Vert a_j \Vert_2^2\]
    where $A_i$ is the $i$th row of $A$ and $a_j$ is $j$th column of $A$. 
    
    Let $C=AB \in \mathbb{C}^{m \times p}$, then
    \[\Vert C\Vert_F^2=\sum_{i=1}^m \sum_{j=1}^p {c_{ij}^2}=\sum_{i=1}^m \sum_{j=1}^p {(A_ib_j)^2}\]
    by the Cauchy-Schwarz inequality, we have
    \begin{align*}
        \sum_{i=1}^m \sum_{j=1}^p {(A_ib_j)^2} 
        &\leq \sum_{i=1}^m \sum_{j=1}^p {\Vert A_i \Vert_2^2 \Vert b_j \Vert_2^2}\\
        &=\left(\sum_{i=1}^m{\Vert A_i \Vert_2^2}\right)\left(\sum_{j=1}^p{\Vert b_j \Vert_2^2}\right)\\
        &=\Vert A \Vert_F^2 \Vert B \Vert_F^2
    \end{align*}
    Hence, we obtain $\Vert AB\Vert_F^2 \leq \Vert A \Vert_F^2 \Vert B \Vert_F^2$, i.e.,$\Vert AB\Vert_F \leq \Vert A \Vert_F \Vert B \Vert_F$
\end{proof}



%                           theorem2.1.7
\begin{lemma}
    Let $U\in \mathbb{C}^{n\times n}$ be unitary matrix, $\mathbf{x}\in \mathbb{C}^n$. Then
    \[\Vert U\mathbf{x} \Vert_2=\Vert \mathbf{x} \Vert_2\]
\end{lemma}

\begin{proof}
     $\Vert U\mathbf{x} \Vert_2^2=\langle U\mathbf{x},U\mathbf{x}\rangle=\langle x,U^*U\mathbf{x}\rangle=\langle \mathbf{x},\mathbf{x}\rangle=\Vert \mathbf{x} \Vert_2^2$, i.e.,$\Vert U\mathbf{x} \Vert_2=\Vert \mathbf{x} \Vert_2$
\end{proof}

theorem2.1.7
%                       induced matrix norm
%我目前只找到 alpha beta=2時的特例
\begin{proof}
    For $\alpha =\beta =2$, let $\mathbf{z}=V\mathbf{x}$
    \begin{align*}
        \Vert UAV \Vert_2
        &=\sup_{\mathbf{x}\neq 0}{\frac{\Vert U(AV\mathbf{x}) \Vert_2}{\Vert \mathbf{x} \Vert_2}}=\sup_{\mathbf{x}\neq 0}{\frac{\Vert AV\mathbf{x} \Vert_2}{\Vert \mathbf{x} \Vert_2}}\\
        &=\sup_{\mathbf{z}\neq 0}{\frac{\Vert A\mathbf{z} \Vert_2}{\Vert V^*\mathbf{z} \Vert_2}}=\sup_{\mathbf{z}\neq 0}{\frac{\Vert A\mathbf{z} \Vert_2}{\Vert \mathbf{z} \Vert_2}}\\
        &=\Vert A \Vert_2
    \end{align*}
\end{proof}

%                               Frobenius norm
\begin{proof}
    \begin{align*}
        \Vert UAV\Vert_F^2
        &=trace((UAV)^*(UAV))\\
        &=trace(V^*A^*U^*UAV)\\
        &=trace(V^*A^*AV)\\
        &=trace(A^*A)\\
        &=\Vert A\Vert_F^2.
    \end{align*}
    Hence, The Frobenius norm is unitarily invariant.
\end{proof}

%                           Example 2.1.9
\begin{proof}
    suppose $rank(A)=r \leq \min{(m,n)}$, by Theorem 1.2.2, we have $A=U \Sigma V^*$, then $A^*A=VDV^*$, 
    \[ D =
    \left[\begin{array}{ccc|ccc}
    \sigma_1^2& & & & & \\
         &\ddots& & & & \\
          & &\sigma_r^2& & & \\ \hline
           & & &0& & \\
            & & & &\ddots& \\
             & & & & &0
    \end{array}
    \right]
    \qquad \text{and} \qquad
    V = 
    \begin{bmatrix}
        | & | &  & | \\
        \mathbf{v}_1 & \mathbf{v}_2 & \cdots & \mathbf{v}_n \\
        | & | &  & |
    \end{bmatrix}.
    \]
    $\{\mathbf{v}_1,\mathbf{v}_2,\cdots ,\mathbf{v}_n\}$ is an orthonormal basis, we sort as $\sigma_1 \geq \sigma_2 \geq \cdots \geq \sigma_r$.\\
    A nonzero $\mathbf{x} \in \mathbb{C}^n$, 
    \[\mathbf{x}=\sum_{i=1}^n{c_iv_i} \quad \text{with}\quad \sum_{i=1}^n{c_i}^2=\Vert \mathbf{x} \Vert_2^2.\]
    Then $\Vert A\mathbf{x} \Vert_2^2=\langle A\mathbf{x},A\mathbf{x}\rangle=\langle A^*A\mathbf{x},\mathbf{x}\rangle$
    \begin{align*}
        \langle A^*A\mathbf{x},\mathbf{x}\rangle
        &=\left\langle\sum_{i=1}^n{c_iA^*Av_i},\sum_{i=1}^n{c_iv_i}\right\rangle\\
        &=\left\langle\sum_{i=1}^n{c_i\sigma_i^2 v_i},\sum_{i=1}^n{c_iv_i}\right\rangle=\sum_{i=1}^n{\sigma_i^2}{c_i}^2\\
        &\leq  \sum_{i=1}^n{\sigma_1^2}{c_i}^2=\sigma_1^2 \sum_{i=1}^n{c_i}^2=\sigma_1^2\Vert \mathbf{x} \Vert_2^2.
    \end{align*}
    Therefore we obtain $\frac{\Vert A\mathbf{x} \Vert_2^2}{\Vert \mathbf{x} \Vert_2^2} \leq \sigma_1^2$, i.e.,$\frac{\Vert A\mathbf{x} \Vert_2}{\Vert \mathbf{x} \Vert_2} \leq \sigma_1$. If we choose $\mathbf{x}=\mathbf{v_1}$, the inequality change into an equality. Hence, we conclude that
    \[\Vert A \Vert_2=\sup_{\mathbf{x} \neq 0}\frac{\Vert A\mathbf{x} \Vert_2}{\Vert \mathbf{x} \Vert_2}=\sigma_1=\sigma_{max}(A)\]    
\end{proof}
%這些是參考資料
%https://www.math.drexel.edu/~foucart/TeachingFiles/F12/M504Lect5.pdf#page2
%https://www.math.drexel.edu/~foucart/TeachingFiles/F12/M504Lect5.pdf#page2
%https://www.sjsu.edu/faculty/guangliang.chen/Math250/lec7matrixnorm.pdf
%https://nhigham.com/2021/02/02/what-is-a-unitarily-invariant-norm/
% Stephen H. Friedberg, Arnold J. Insel, Lawrence E. Spence. Linear Algebra, 4th edition. Pearson Education, Upper Saddle River, New Jersey.
\end{document}